\section{Résultats obtenus}

\subsection{Descentes}
\subsubsection{Résultats des versus}
Les deux tableaux suivants présentent combien de fois sur 10 exécution une méthode gagne contre l'autre. 

Pour methode1\_vs\_methode2, $(n_1,n_2)$ signifie que methode1 gagne $n_1$ fois et methode2 gagne $n_2$ fois. Évidemment $n_1+n_2=10$.
\iffalse
\begin{table}[ht]
	\rowcolors{2}{gray!10}{white}
	\centering
	\caption{list of symbols}
	\begin{tabular}[t]
		{m{0.1\textwidth}m{0.25\textwidth}m{0.25\textwidth}m{0.2\textwidth}}
		\toprule
		\textbf{Symbol}&\textbf{dimension}&\textbf{Unit}&\textbf{Unit abbreviation}\\
		\midrule
		1&2&3&4\\
		1&2&3&4\\
		1&2&3&4\\
		1&2&3&4\\
		\bottomrule
	\end{tabular}

\end{table}


\begin{table}[ht]
	\rowcolors{2}{gray!10}{white}
	\centering
	\caption{Nombre de fois qu'une méthode gagne contre l'autre}
		\begin{tabular}[t]
			{m{0.2\textwidth}m{0.2\textwidth}m{0.2\textwidth}m{0.2\textwidth} m{0.2\textwidth} m{0.2\textwidth}m{0.2\textwidth}}
			\toprule
	instances &	swap\_first\_vs\_best&	swap\_first\_vs\_worst &	swap\_best\_vs\_worst&	2opt\_first\_vs\_best&	2opt\_first\_vs\_worst&	2opt\_best\_vs\_worst\\
	\midrule
	rand\_50\_75.txt&	(6, 4)&	(5, 5)&	(4, 6)&	(7, 3)&	(8, 2)&	(6, 4)\\
	rand\_50\_100.txt&	(5, 5)&	(6, 4)&	(5, 5)&	(9, 1)&	(10, 0)&	(7, 3)\\
	rand\_60\_50.txt&	(4, 6)&	(3, 7)&	(5, 5)&	(8, 2)&	(10, 0)&	(7, 3)\\
	rand\_60\_100.txt&	(4, 6)&	(5, 5)&	(5, 5)&	(9, 1)&	(10, 0)&	(7, 3)\\
	rand\_70\_55.txt&	(7, 3)&	(4, 6)&	(5, 5)&	(10, 0)&	(8, 2)&	(3, 7)\\
	rand\_70\_100.txt&	(5, 5)&	(6, 4)&	(4, 6)&	(9, 1)&	(10, 0)&	(5, 5)\\
	rand\_80\_70.txt&	(3, 7)&	(5, 5)&	(3, 7)&	(10, 0)&	(10, 0)&	(5, 5)\\
	rand\_80\_100.txt&	(6, 4)&	(5, 5)&	(5, 5)&	(9, 1)	&(10, 0)&	(4, 6)\\
	\bottomrule
\end{tabular}
\end{table}
\fi
\begin{table}[ht]
	\rowcolors{2}{gray!10}{white}
	\centering
	\caption{Nombre de fois qu'une méthode gagne contre l'autre (swap)}
	\begin{tabular}[t]
		{m{0.2\textwidth}m{0.2\textwidth}m{0.2\textwidth}m{0.2\textwidth}}
		\toprule
		instances &	swap\_first\_vs\_best&	swap\_first\_vs\_worst &	swap\_best\_vs\_worst\\
		\midrule
		rand\_50\_75.txt&	(6, 4)&	(5, 5)&	(4, 6)\\
		rand\_50\_100.txt&	(5, 5)&	(6, 4)&	(5, 5)\\
		rand\_60\_50.txt&	(4, 6)&	(3, 7)&	(5, 5)\\
		rand\_60\_100.txt&	(4, 6)&	(5, 5)&	(5, 5)\\
		rand\_70\_55.txt&	(7, 3)&	(4, 6)&	(5, 5)\\
		rand\_70\_100.txt&	(5, 5)&	(6, 4)&	(4, 6)\\
		rand\_80\_70.txt&	(3, 7)&	(5, 5)&	(3, 7)\\
		rand\_80\_100.txt&	(6, 4)&	(5, 5)&	(5, 5)\\
		\bottomrule
	\end{tabular}
\end{table}

\begin{table}[ht]
	\rowcolors{2}{gray!10}{white}
	\centering
	\caption{Nombre de fois qu'une méthode gagne contre l'autre (2opt)}
	\begin{tabular}[t]
		{m{0.2\textwidth}m{0.2\textwidth}m{0.2\textwidth}m{0.2\textwidth} m{0.2\textwidth} m{0.2\textwidth}m{0.2\textwidth}}
		\toprule
		instances &		2opt\_first\_vs\_best&	2opt\_first\_vs\_worst&	2opt\_best\_vs\_worst\\
		\midrule
		rand\_50\_75.txt&		(7, 3)&	(8, 2)&	(6, 4)\\
		rand\_50\_100.txt&		(9, 1)&	(10, 0)&	(7, 3)\\
		rand\_60\_50.txt&		(8, 2)&	(10, 0)&	(7, 3)\\
		rand\_60\_100.txt&		(9, 1)&	(10, 0)&	(7, 3)\\
		rand\_70\_55.txt&		(10, 0)&	(8, 2)&	(3, 7)\\
		rand\_70\_100.txt&		(9, 1)&	(10, 0)&	(5, 5)\\
		rand\_80\_70.txt&	    (10, 0)&	(10, 0)&	(5, 5)\\
		rand\_80\_100.txt&		(9, 1)	&(10, 0)&	(4, 6)\\
		\bottomrule
	\end{tabular}
\end{table}

\subsubsection{Moyenne}
En prenant la moyenne des scores, on a :  

\begin{table}[ht]
	\rowcolors{2}{gray!10}{white}
	\centering
	\caption{Moyenne des scores}
	\begin{tabular}[t]
		{m{0.2\textwidth}m{0.2\textwidth}m{0.2\textwidth}m{0.2\textwidth} }
		\toprule
		 &		first&	worst&	best\\
		\midrule
		swap&		1205.8250&	1207.7250&	1206.7625\\
		2opt&		1596.2125&1809.1750 &	1801.5000\\
	
		\bottomrule
	\end{tabular}
\end{table}

On constate que dans le deux cas, \textbf{first} semble être la meilleur méthode.

%On peut aussi déduire des deux premiers tableaux que\textbf{ first} gagne \textbf{115} fois contre \textbf{worst} et\textbf{ worst} seulement \textbf{45} fois contre \textbf{first}. 

\subsection{ILS et SW}

\subsubsection{ILS}
\begin{itemize}
	\item \textbf{Résultats des versus}
	
	Le tableau suivant présente combien de fois sur 20 exécutions (deux valeurs pour nombre de perturbations) une méthode gagne contre l'autre. 
	
	\begin{table}[ht]
		\rowcolors{2}{gray!10}{white}
		\centering
		\caption{Nombre de fois qu'une méthode gagne contre l'autre}
		\begin{tabular}[t]
			{m{0.2\textwidth}m{0.2\textwidth}m{0.2\textwidth}}
			\toprule
			instances &	swap\_first\_vs\_best &	2opt\_first\_vs\_best \\
			
			\midrule
			rand\_50\_75.txt&	(11, 9)	&(16, 4)\\
			rand\_50\_100.txt&	(10, 10)&	(19, 1)\\
			rand\_60\_50.txt&	(13, 7)&	(17, 3)\\
			rand\_60\_100.txt&	(5, 15)&	(15, 5)\\
			rand\_70\_55.txt&	(14, 6)&	(16, 4)\\
			rand\_70\_100.txt&	(13, 7)	&(19, 1)\\
			rand\_80\_70.txt&	(14, 6)&	(18, 2)\\
			rand\_80\_100.txt&	(19, 1)&	(17, 3)\\
			
		
			\bottomrule
		\end{tabular}
	\end{table}
\item \textbf{Moyenne }

\begin{table}[ht]
	\rowcolors{2}{gray!10}{white}
	\centering
	\caption{Moyenne des scores}
	\begin{tabular}[t]
		{m{0.2\textwidth}m{0.2\textwidth}m{0.2\textwidth}}
		\toprule
		méthode &		first&		best\\
		\midrule
		swap&		1166.0875&	1180.1875\\
		2opt&		1553.9625& 1697.6750 \\
		
		\bottomrule
	\end{tabular}
\end{table}
On constate que \textbf{first} domine \textbf{best}.
\end{itemize}


\subsubsection{SW}
\begin{itemize}
	\item \textbf{Résultats des versus}
	
	Le tableau suivant présente combien de fois sur 20 exécutions (deux valeurs pour nombre de perturbations) une méthode gagne contre l'autre. 
	
	\begin{table}[ht]
		\rowcolors{2}{gray!10}{white}
		\centering
		\caption{Nombre de fois qu'une méthode gagne contre l'autre}
		\begin{tabular}[t]
			{m{0.2\textwidth}m{0.2\textwidth}}
			\toprule
			instances &	swap\_vs\_2opt  \\
			
			\midrule
			rand\_50\_75.txt& (10, 10)	\\
			rand\_50\_100.txt& (13, 7)	\\
			rand\_60\_50.txt& (10, 10)	\\
			rand\_60\_100.txt& (8, 12)	\\
			rand\_70\_55.txt&	(10, 10)\\
			rand\_70\_100.txt&	(9, 11)\\
			rand\_80\_70.txt&(12, 8)	\\
			rand\_80\_100.txt&	(6, 14)\\
			
			
			\bottomrule
		\end{tabular}
	\end{table}
	\item \textbf{Moyenne }
	
	\begin{table}[ht]
		\rowcolors{2}{gray!10}{white}
		\centering
		\caption{Moyenne des scores}
		\begin{tabular}[t]
			{m{0.2\textwidth}m{0.2\textwidth}m{0.2\textwidth}}
			\toprule
			 &		Moyenne\\
			\midrule
			swap&		2861.76875 \\
			2opt&		2829.88125\\
			
			\bottomrule
		\end{tabular}
	\end{table}
	On constate que \textbf{2opt} domine \textbf{swap}.
\end{itemize}