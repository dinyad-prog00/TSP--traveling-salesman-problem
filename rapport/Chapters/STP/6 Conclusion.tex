\section{Conclusion}
L'analyse des performances des descentes, des algorithmes de recherche locale moins strictes, a fourni des résultats significatifs. Voici quelques observations clés :

\begin{itemize}
	\item La descente \textbf{first} s'est avérée plus rapide et plus performante que \textbf{best} et \textbf{worst} dans l'ensemble.


\item La relation de voisinage \textbf{swap} a montré une performance parfois supérieure à celle de \textbf{2-opt}.


\item L'introduction de l'évaluation incrémentale a considérablement amélioré la rapidité des algorithmes en réduisant le nombre d'évaluations nécessaires. L'évaluation incrémentale offre une approche efficace pour économiser des ressources.

\item Dans le contexte d'ILS, la stratégie \textbf{first} a aussi montré une domination par rapport à d'autres approches.


\item L'algorithmes génétique a montré de bons résultats en termes de qualité des solutions obtenues. L'efficacité de l'algorithme génétique peut être influencée par la taille de la population, la probabilité de croisement, et d'autres paramètres.

\end{itemize}
En conclusion, il n'y a pas d'approche unique qui fonctionne le mieux dans tous les cas. Le choix de l'algorithme et des heuristiques dépend fortement des caractéristiques spécifiques du problème. Une approche expérimentale peut aider à identifier les stratégies les plus performantes dans des contextes variés. Les résultats obtenus suggèrent que des techniques telles que la descente \textbf{first}, le voisinage \textbf{swap},  peuvent être des choix prometteurs selon le contexte du problème étudié.
\newline

Le code est aussi disponible sur mon github :
\href{}{https://github.com/dinyad-prog00/TSP--traveling-salesman-problem} 


