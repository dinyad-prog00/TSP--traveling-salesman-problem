\section{Présentation du problème}

Le problème du voyageur de commerce \cite{wiki-tsp} peut être défini comme suit :

Soit $G=(V,E)$ un graphe complet, où $V$ est l'ensemble des villes et $E$ est l'ensemble des arêtes pondérées représentant les distances entre les villes. L'objectif est de trouver un cycle hamiltonien $C$ qui visite chaque ville une fois et minimise la somme des distances parcourues.
\newline

Mathématiquement, cela peut être exprimé comme la recherche de la permutation $\pi$ pour minimiser:

$$ d_{Total} = \sum_{i=1}^{n-1}d(\pi(i),\pi(i+1))+d(\pi(n),\pi(1)))$$

où $d(u,v)$) représente la distance entre les villes $u$ et $v$.
